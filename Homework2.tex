% Options for packages loaded elsewhere
\PassOptionsToPackage{unicode}{hyperref}
\PassOptionsToPackage{hyphens}{url}
%
\documentclass[
]{article}
\usepackage{amsmath,amssymb}
\usepackage{lmodern}
\usepackage{iftex}
\ifPDFTeX
  \usepackage[T1]{fontenc}
  \usepackage[utf8]{inputenc}
  \usepackage{textcomp} % provide euro and other symbols
\else % if luatex or xetex
  \usepackage{unicode-math}
  \defaultfontfeatures{Scale=MatchLowercase}
  \defaultfontfeatures[\rmfamily]{Ligatures=TeX,Scale=1}
\fi
% Use upquote if available, for straight quotes in verbatim environments
\IfFileExists{upquote.sty}{\usepackage{upquote}}{}
\IfFileExists{microtype.sty}{% use microtype if available
  \usepackage[]{microtype}
  \UseMicrotypeSet[protrusion]{basicmath} % disable protrusion for tt fonts
}{}
\makeatletter
\@ifundefined{KOMAClassName}{% if non-KOMA class
  \IfFileExists{parskip.sty}{%
    \usepackage{parskip}
  }{% else
    \setlength{\parindent}{0pt}
    \setlength{\parskip}{6pt plus 2pt minus 1pt}}
}{% if KOMA class
  \KOMAoptions{parskip=half}}
\makeatother
\usepackage{xcolor}
\IfFileExists{xurl.sty}{\usepackage{xurl}}{} % add URL line breaks if available
\IfFileExists{bookmark.sty}{\usepackage{bookmark}}{\usepackage{hyperref}}
\hypersetup{
  pdftitle={Домашна работа 2 по R},
  pdfauthor={Николай Желязков, 82022},
  hidelinks,
  pdfcreator={LaTeX via pandoc}}
\urlstyle{same} % disable monospaced font for URLs
\usepackage[margin=1in]{geometry}
\usepackage{color}
\usepackage{fancyvrb}
\newcommand{\VerbBar}{|}
\newcommand{\VERB}{\Verb[commandchars=\\\{\}]}
\DefineVerbatimEnvironment{Highlighting}{Verbatim}{commandchars=\\\{\}}
% Add ',fontsize=\small' for more characters per line
\usepackage{framed}
\definecolor{shadecolor}{RGB}{248,248,248}
\newenvironment{Shaded}{\begin{snugshade}}{\end{snugshade}}
\newcommand{\AlertTok}[1]{\textcolor[rgb]{0.94,0.16,0.16}{#1}}
\newcommand{\AnnotationTok}[1]{\textcolor[rgb]{0.56,0.35,0.01}{\textbf{\textit{#1}}}}
\newcommand{\AttributeTok}[1]{\textcolor[rgb]{0.77,0.63,0.00}{#1}}
\newcommand{\BaseNTok}[1]{\textcolor[rgb]{0.00,0.00,0.81}{#1}}
\newcommand{\BuiltInTok}[1]{#1}
\newcommand{\CharTok}[1]{\textcolor[rgb]{0.31,0.60,0.02}{#1}}
\newcommand{\CommentTok}[1]{\textcolor[rgb]{0.56,0.35,0.01}{\textit{#1}}}
\newcommand{\CommentVarTok}[1]{\textcolor[rgb]{0.56,0.35,0.01}{\textbf{\textit{#1}}}}
\newcommand{\ConstantTok}[1]{\textcolor[rgb]{0.00,0.00,0.00}{#1}}
\newcommand{\ControlFlowTok}[1]{\textcolor[rgb]{0.13,0.29,0.53}{\textbf{#1}}}
\newcommand{\DataTypeTok}[1]{\textcolor[rgb]{0.13,0.29,0.53}{#1}}
\newcommand{\DecValTok}[1]{\textcolor[rgb]{0.00,0.00,0.81}{#1}}
\newcommand{\DocumentationTok}[1]{\textcolor[rgb]{0.56,0.35,0.01}{\textbf{\textit{#1}}}}
\newcommand{\ErrorTok}[1]{\textcolor[rgb]{0.64,0.00,0.00}{\textbf{#1}}}
\newcommand{\ExtensionTok}[1]{#1}
\newcommand{\FloatTok}[1]{\textcolor[rgb]{0.00,0.00,0.81}{#1}}
\newcommand{\FunctionTok}[1]{\textcolor[rgb]{0.00,0.00,0.00}{#1}}
\newcommand{\ImportTok}[1]{#1}
\newcommand{\InformationTok}[1]{\textcolor[rgb]{0.56,0.35,0.01}{\textbf{\textit{#1}}}}
\newcommand{\KeywordTok}[1]{\textcolor[rgb]{0.13,0.29,0.53}{\textbf{#1}}}
\newcommand{\NormalTok}[1]{#1}
\newcommand{\OperatorTok}[1]{\textcolor[rgb]{0.81,0.36,0.00}{\textbf{#1}}}
\newcommand{\OtherTok}[1]{\textcolor[rgb]{0.56,0.35,0.01}{#1}}
\newcommand{\PreprocessorTok}[1]{\textcolor[rgb]{0.56,0.35,0.01}{\textit{#1}}}
\newcommand{\RegionMarkerTok}[1]{#1}
\newcommand{\SpecialCharTok}[1]{\textcolor[rgb]{0.00,0.00,0.00}{#1}}
\newcommand{\SpecialStringTok}[1]{\textcolor[rgb]{0.31,0.60,0.02}{#1}}
\newcommand{\StringTok}[1]{\textcolor[rgb]{0.31,0.60,0.02}{#1}}
\newcommand{\VariableTok}[1]{\textcolor[rgb]{0.00,0.00,0.00}{#1}}
\newcommand{\VerbatimStringTok}[1]{\textcolor[rgb]{0.31,0.60,0.02}{#1}}
\newcommand{\WarningTok}[1]{\textcolor[rgb]{0.56,0.35,0.01}{\textbf{\textit{#1}}}}
\usepackage{graphicx}
\makeatletter
\def\maxwidth{\ifdim\Gin@nat@width>\linewidth\linewidth\else\Gin@nat@width\fi}
\def\maxheight{\ifdim\Gin@nat@height>\textheight\textheight\else\Gin@nat@height\fi}
\makeatother
% Scale images if necessary, so that they will not overflow the page
% margins by default, and it is still possible to overwrite the defaults
% using explicit options in \includegraphics[width, height, ...]{}
\setkeys{Gin}{width=\maxwidth,height=\maxheight,keepaspectratio}
% Set default figure placement to htbp
\makeatletter
\def\fps@figure{htbp}
\makeatother
\setlength{\emergencystretch}{3em} % prevent overfull lines
\providecommand{\tightlist}{%
  \setlength{\itemsep}{0pt}\setlength{\parskip}{0pt}}
\setcounter{secnumdepth}{-\maxdimen} % remove section numbering
\ifLuaTeX
  \usepackage{selnolig}  % disable illegal ligatures
\fi

\title{Домашна работа 2 по R}
\author{Николай Желязков, 82022}
\date{}

\begin{document}
\maketitle

\hypertarget{ux437ux430ux434ux430ux447ux430-1}{%
\subsection{\texorpdfstring{\textbf{Задача
1}}{Задача 1}}\label{ux437ux430ux434ux430ux447ux430-1}}

\begin{Shaded}
\begin{Highlighting}[]
  \CommentTok{\#Проверяваме дали отхвърля H0}
\NormalTok{denyZeroHyp }\OtherTok{=} \ControlFlowTok{function}\NormalTok{(n)\{}
\NormalTok{  t }\OtherTok{=} \FunctionTok{table}\NormalTok{(}\FunctionTok{sample}\NormalTok{(}\DecValTok{1}\SpecialCharTok{:}\DecValTok{6}\NormalTok{,}\AttributeTok{size =}\NormalTok{ n,}\AttributeTok{replace =}\NormalTok{ T))}
\NormalTok{  result }\OtherTok{=} \FunctionTok{chisq.test}\NormalTok{(t)}\SpecialCharTok{$}\NormalTok{p.value}\SpecialCharTok{\textless{}}\FloatTok{0.05}
\NormalTok{  result}
\NormalTok{\}}
\end{Highlighting}
\end{Shaded}

\begin{Shaded}
\begin{Highlighting}[]
\CommentTok{\#Гледаме какъв процент от всички отхъврлят H0}
\NormalTok{f1 }\OtherTok{=} \ControlFlowTok{function}\NormalTok{(n)\{}
\NormalTok{  s }\OtherTok{=} \FunctionTok{sum}\NormalTok{(}\FunctionTok{replicate}\NormalTok{(}\DecValTok{10000}\NormalTok{,}\FunctionTok{denyZeroHyp}\NormalTok{(n)))}\SpecialCharTok{/}\DecValTok{10000}\SpecialCharTok{*}\DecValTok{100}
\NormalTok{  result }\OtherTok{=} \FunctionTok{paste}\NormalTok{(s, }\StringTok{"\%"}\NormalTok{)}
\NormalTok{  result}
\NormalTok{\}}
\FunctionTok{f1}\NormalTok{(}\DecValTok{100}\NormalTok{)}
\end{Highlighting}
\end{Shaded}

\begin{verbatim}
## [1] "4.55 %"
\end{verbatim}

\begin{Shaded}
\begin{Highlighting}[]
\FunctionTok{f1}\NormalTok{(}\DecValTok{200}\NormalTok{)}
\end{Highlighting}
\end{Shaded}

\begin{verbatim}
## [1] "5.17 %"
\end{verbatim}

\begin{Shaded}
\begin{Highlighting}[]
\FunctionTok{f1}\NormalTok{(}\DecValTok{400}\NormalTok{)}
\end{Highlighting}
\end{Shaded}

\begin{verbatim}
## [1] "4.72 %"
\end{verbatim}

\hypertarget{ux437ux430ux434ux430ux447ux430-2}{%
\subsection{\texorpdfstring{\textbf{Задача
2}}{Задача 2}}\label{ux437ux430ux434ux430ux447ux430-2}}

\begin{Shaded}
\begin{Highlighting}[]
\CommentTok{\#Проверяваме дали данните са от нормално разпределение}
\NormalTok{checkNormality }\OtherTok{=} \ControlFlowTok{function}\NormalTok{(n) \{}
\NormalTok{    x }\OtherTok{=} \FunctionTok{runif}\NormalTok{(n, }\DecValTok{5}\NormalTok{, }\DecValTok{12}\NormalTok{)}
\NormalTok{    hypResults }\OtherTok{=} \FunctionTok{shapiro.test}\NormalTok{(x)}\SpecialCharTok{$}\NormalTok{p.value }\SpecialCharTok{\textgreater{}} \FloatTok{0.05}
\NormalTok{    hypResults}
\NormalTok{\}}
\end{Highlighting}
\end{Shaded}

\begin{Shaded}
\begin{Highlighting}[]
\CommentTok{\#Намираме  Колко често заключението на теста е вярно }
\NormalTok{f2 }\OtherTok{=} \ControlFlowTok{function}\NormalTok{(nBig,n)\{}
\NormalTok{  results }\OtherTok{=} \FunctionTok{replicate}\NormalTok{(}\AttributeTok{n=}\NormalTok{nBig,}\FunctionTok{checkNormality}\NormalTok{(n))}
  \FunctionTok{prop.table}\NormalTok{(}\FunctionTok{table}\NormalTok{(results))}
\NormalTok{\}}
\FunctionTok{f2}\NormalTok{(}\DecValTok{10000}\NormalTok{,}\DecValTok{15}\NormalTok{)}
\end{Highlighting}
\end{Shaded}

\begin{verbatim}
## results
##  FALSE   TRUE 
## 0.1256 0.8744
\end{verbatim}

\begin{Shaded}
\begin{Highlighting}[]
\FunctionTok{f2}\NormalTok{(}\DecValTok{10000}\NormalTok{,}\DecValTok{25}\NormalTok{)}
\end{Highlighting}
\end{Shaded}

\begin{verbatim}
## results
##  FALSE   TRUE 
## 0.2916 0.7084
\end{verbatim}

\begin{Shaded}
\begin{Highlighting}[]
\FunctionTok{f2}\NormalTok{(}\DecValTok{10000}\NormalTok{,}\DecValTok{50}\NormalTok{)}
\end{Highlighting}
\end{Shaded}

\begin{verbatim}
## results
##  FALSE   TRUE 
## 0.7501 0.2499
\end{verbatim}

\begin{Shaded}
\begin{Highlighting}[]
\FunctionTok{f2}\NormalTok{(}\DecValTok{10000}\NormalTok{,}\DecValTok{90}\NormalTok{)}
\end{Highlighting}
\end{Shaded}

\begin{verbatim}
## results
##  FALSE   TRUE 
## 0.9883 0.0117
\end{verbatim}

\hypertarget{ux437ux430ux434ux430ux447ux430-3}{%
\subsection{\texorpdfstring{\textbf{Задача
3}}{Задача 3}}\label{ux437ux430ux434ux430ux447ux430-3}}

\begin{Shaded}
\begin{Highlighting}[]
\CommentTok{\#Функция четаеща графиката на оценката на бета1}
\NormalTok{graphics }\OtherTok{=} \ControlFlowTok{function}\NormalTok{(beta1\_est)\{}
  \FunctionTok{hist}\NormalTok{(beta1\_est)}
  \FunctionTok{qqnorm}\NormalTok{(beta1\_est)}
\NormalTok{\}}
\end{Highlighting}
\end{Shaded}

\begin{Shaded}
\begin{Highlighting}[]
\NormalTok{f3 }\OtherTok{=} \ControlFlowTok{function}\NormalTok{(n)\{}
\NormalTok{  beta1 }\OtherTok{=} \DecValTok{5}
  \CommentTok{\# 4 реда {-} по 1 за всяка подточка}
\NormalTok{  beta1\_est }\OtherTok{=}  \FunctionTok{matrix}\NormalTok{(}\AttributeTok{data =} \ConstantTok{NA}\NormalTok{, }\AttributeTok{nrow =} \DecValTok{4}\NormalTok{, }\AttributeTok{ncol =} \DecValTok{10000}\NormalTok{)}
\NormalTok{  beta1\_sd\_err }\OtherTok{=} \FunctionTok{matrix}\NormalTok{(}\AttributeTok{data =} \ConstantTok{NA}\NormalTok{, }\AttributeTok{nrow =} \DecValTok{4}\NormalTok{, }\AttributeTok{ncol =} \DecValTok{10000}\NormalTok{)}
\NormalTok{  beta1\_sd\_err }\OtherTok{=} \FunctionTok{matrix}\NormalTok{(}\AttributeTok{data =} \ConstantTok{NA}\NormalTok{, }\AttributeTok{nrow =} \DecValTok{4}\NormalTok{, }\AttributeTok{ncol =} \DecValTok{10000}\NormalTok{)}
\NormalTok{  leftConfInt }\OtherTok{=} \FunctionTok{matrix}\NormalTok{(}\AttributeTok{data =} \ConstantTok{NA}\NormalTok{, }\AttributeTok{nrow =} \DecValTok{4}\NormalTok{, }\AttributeTok{ncol =} \DecValTok{10000}\NormalTok{)}
\NormalTok{  rightConfInt }\OtherTok{=} \FunctionTok{matrix}\NormalTok{(}\AttributeTok{data =} \ConstantTok{NA}\NormalTok{, }\AttributeTok{nrow =} \DecValTok{4}\NormalTok{, }\AttributeTok{ncol =} \DecValTok{10000}\NormalTok{)}
  
  \CommentTok{\#генерираме данните 10000 пъти}
  \ControlFlowTok{for}\NormalTok{ (k }\ControlFlowTok{in} \DecValTok{1}\SpecialCharTok{:}\DecValTok{10000}\NormalTok{) \{}
\NormalTok{    x }\OtherTok{=} \FunctionTok{sample}\NormalTok{(}\DecValTok{1}\SpecialCharTok{:}\DecValTok{10}\NormalTok{, }\AttributeTok{size =}\NormalTok{ n, }\AttributeTok{replace =} \ConstantTok{TRUE}\NormalTok{)}
    
    \CommentTok{\# а)}
\NormalTok{    e }\OtherTok{=} \FunctionTok{rnorm}\NormalTok{(n, }\AttributeTok{mean =} \DecValTok{0}\NormalTok{, }\AttributeTok{sd =} \DecValTok{2}\NormalTok{)}
\NormalTok{    y }\OtherTok{=}\NormalTok{ beta1 }\SpecialCharTok{*}\NormalTok{ x }\SpecialCharTok{+}\NormalTok{ e}
    
\NormalTok{    df }\OtherTok{=} \FunctionTok{data.frame}\NormalTok{(x, y)}
\NormalTok{    model }\OtherTok{=} \FunctionTok{lm}\NormalTok{(y }\SpecialCharTok{\textasciitilde{}}\NormalTok{ x, }\AttributeTok{data =}\NormalTok{ df)}
\NormalTok{    model\_summary }\OtherTok{=} \FunctionTok{summary}\NormalTok{(model)}
   
\NormalTok{    beta1\_est[}\DecValTok{1}\NormalTok{, k] }\OtherTok{=}\NormalTok{ model\_summary}\SpecialCharTok{$}\NormalTok{coefficients[}\DecValTok{2}\NormalTok{, }\DecValTok{1}\NormalTok{] }\CommentTok{\# оценка на бета1}
\NormalTok{    beta1\_sd\_err[}\DecValTok{1}\NormalTok{, k] }\OtherTok{=}\NormalTok{ model\_summary}\SpecialCharTok{$}\NormalTok{coefficients[}\DecValTok{2}\NormalTok{, }\DecValTok{2}\NormalTok{]}
\NormalTok{    leftConfInt[}\DecValTok{1}\NormalTok{,k] }\OtherTok{=} \FunctionTok{confint}\NormalTok{(model)[}\DecValTok{2}\NormalTok{,}\DecValTok{1}\NormalTok{] }\CommentTok{\#левия край на дов. интервал за бета1}
\NormalTok{    rightConfInt[}\DecValTok{1}\NormalTok{,k] }\OtherTok{=} \FunctionTok{confint}\NormalTok{(model)[}\DecValTok{2}\NormalTok{,}\DecValTok{2}\NormalTok{] }\CommentTok{\#десния край на дов. интервал за бета1}
    
    \CommentTok{\# б)}
\NormalTok{    e }\OtherTok{=} \FunctionTok{runif}\NormalTok{(n, }\SpecialCharTok{{-}}\FloatTok{3.5}\NormalTok{, }\FloatTok{3.5}\NormalTok{)}
\NormalTok{    y }\OtherTok{=}\NormalTok{ beta1 }\SpecialCharTok{*}\NormalTok{ x }\SpecialCharTok{+}\NormalTok{ e}
    
\NormalTok{    df }\OtherTok{=} \FunctionTok{data.frame}\NormalTok{(x, y)}
\NormalTok{    model }\OtherTok{=} \FunctionTok{lm}\NormalTok{(y }\SpecialCharTok{\textasciitilde{}}\NormalTok{ x, }\AttributeTok{data =}\NormalTok{ df)}
\NormalTok{    model\_summary }\OtherTok{=} \FunctionTok{summary}\NormalTok{(model)}
\NormalTok{    p }\OtherTok{=} \FunctionTok{paste}\NormalTok{(}\StringTok{"b)"}\NormalTok{)}
\NormalTok{    beta1\_est[}\DecValTok{2}\NormalTok{, k] }\OtherTok{=}\NormalTok{ model\_summary}\SpecialCharTok{$}\NormalTok{coefficients[}\DecValTok{2}\NormalTok{, }\DecValTok{1}\NormalTok{] }\CommentTok{\# оценка на бета1}
\NormalTok{    beta1\_sd\_err[}\DecValTok{2}\NormalTok{, k] }\OtherTok{=}\NormalTok{ model\_summary}\SpecialCharTok{$}\NormalTok{coefficients[}\DecValTok{2}\NormalTok{, }\DecValTok{2}\NormalTok{]}
\NormalTok{    leftConfInt[}\DecValTok{2}\NormalTok{,k] }\OtherTok{=} \FunctionTok{confint}\NormalTok{(model)[}\DecValTok{2}\NormalTok{,}\DecValTok{1}\NormalTok{] }\CommentTok{\#левия край на дов. интервал за бета1}
\NormalTok{    rightConfInt[}\DecValTok{2}\NormalTok{,k] }\OtherTok{=} \FunctionTok{confint}\NormalTok{(model)[}\DecValTok{2}\NormalTok{,}\DecValTok{2}\NormalTok{] }\CommentTok{\#десния край на дов. интервал за бета1}
    
    \CommentTok{\# в)}
\NormalTok{    v }\OtherTok{=} \FunctionTok{rexp}\NormalTok{(n, }\FloatTok{0.7}\NormalTok{)}
\NormalTok{    w }\OtherTok{=} \FunctionTok{rexp}\NormalTok{(n, }\FloatTok{0.7}\NormalTok{)}
\NormalTok{    e }\OtherTok{=}\NormalTok{ v }\SpecialCharTok{{-}}\NormalTok{ w}
\NormalTok{    y }\OtherTok{=}\NormalTok{ beta1 }\SpecialCharTok{*}\NormalTok{ x }\SpecialCharTok{+}\NormalTok{ e}
    
\NormalTok{    df }\OtherTok{=} \FunctionTok{data.frame}\NormalTok{(x, y)}
\NormalTok{    model }\OtherTok{=} \FunctionTok{lm}\NormalTok{(y }\SpecialCharTok{\textasciitilde{}}\NormalTok{ x, }\AttributeTok{data =}\NormalTok{ df)}
\NormalTok{    model\_summary }\OtherTok{=} \FunctionTok{summary}\NormalTok{(model)}
\NormalTok{    beta1\_est[}\DecValTok{3}\NormalTok{, k] }\OtherTok{=}\NormalTok{ model\_summary}\SpecialCharTok{$}\NormalTok{coefficients[}\DecValTok{2}\NormalTok{, }\DecValTok{1}\NormalTok{] }\CommentTok{\# оценка на бета1}
\NormalTok{    beta1\_sd\_err[}\DecValTok{3}\NormalTok{, k] }\OtherTok{=}\NormalTok{ model\_summary}\SpecialCharTok{$}\NormalTok{coefficients[}\DecValTok{2}\NormalTok{, }\DecValTok{2}\NormalTok{]}
\NormalTok{    leftConfInt[}\DecValTok{3}\NormalTok{,k] }\OtherTok{=} \FunctionTok{confint}\NormalTok{(model)[}\DecValTok{2}\NormalTok{,}\DecValTok{1}\NormalTok{] }\CommentTok{\#левия край на дов. интервал за бета1}
\NormalTok{    rightConfInt[}\DecValTok{3}\NormalTok{,k] }\OtherTok{=} \FunctionTok{confint}\NormalTok{(model)[}\DecValTok{2}\NormalTok{,}\DecValTok{2}\NormalTok{]  }\CommentTok{\#десния край на дов. интервал за бета1}
    
    \CommentTok{\# г)}
\NormalTok{    u }\OtherTok{=} \FunctionTok{rexp}\NormalTok{(n, }\FloatTok{0.5}\NormalTok{)}
\NormalTok{    e }\OtherTok{=}\NormalTok{ u }\SpecialCharTok{{-}} \DecValTok{2}
\NormalTok{    y }\OtherTok{=}\NormalTok{ beta1 }\SpecialCharTok{*}\NormalTok{ x }\SpecialCharTok{+}\NormalTok{ e}
    
\NormalTok{    df }\OtherTok{=} \FunctionTok{data.frame}\NormalTok{(x, y)}
\NormalTok{    model }\OtherTok{=} \FunctionTok{lm}\NormalTok{(y }\SpecialCharTok{\textasciitilde{}}\NormalTok{ x, }\AttributeTok{data =}\NormalTok{ df)}
\NormalTok{    model\_summary }\OtherTok{=} \FunctionTok{summary}\NormalTok{(model)}
 
\NormalTok{    beta1\_est[}\DecValTok{4}\NormalTok{, k] }\OtherTok{=}\NormalTok{ model\_summary}\SpecialCharTok{$}\NormalTok{coefficients[}\DecValTok{2}\NormalTok{, }\DecValTok{1}\NormalTok{] }\CommentTok{\# оценка на бета1}
\NormalTok{    beta1\_sd\_err[}\DecValTok{4}\NormalTok{, k] }\OtherTok{=}\NormalTok{ model\_summary}\SpecialCharTok{$}\NormalTok{coefficients[}\DecValTok{2}\NormalTok{, }\DecValTok{2}\NormalTok{] }
\NormalTok{    leftConfInt[}\DecValTok{4}\NormalTok{,k] }\OtherTok{=} \FunctionTok{confint}\NormalTok{(model)[}\DecValTok{2}\NormalTok{,}\DecValTok{1}\NormalTok{] }\CommentTok{\#левия край на дов. интервал за бета1}
\NormalTok{    rightConfInt[}\DecValTok{4}\NormalTok{,k] }\OtherTok{=} \FunctionTok{confint}\NormalTok{(model)[}\DecValTok{2}\NormalTok{,}\DecValTok{2}\NormalTok{]  }\CommentTok{\#десния край на дов. интервал за бета1}
\NormalTok{  \}}
   \CommentTok{\#Създаваме матрица за крайните резултати}
\NormalTok{  tbl }\OtherTok{=} \FunctionTok{matrix}\NormalTok{(}\AttributeTok{data =} \ConstantTok{NA}\NormalTok{, }\AttributeTok{nrow =} \DecValTok{4}\NormalTok{, }\AttributeTok{ncol =} \DecValTok{3}\NormalTok{)}
  \FunctionTok{rownames}\NormalTok{(tbl) }\OtherTok{=} \FunctionTok{c}\NormalTok{(}\StringTok{"а)"}\NormalTok{, }\StringTok{"б)"}\NormalTok{,}\StringTok{"в)"}\NormalTok{,}\StringTok{"г)"}\NormalTok{)}
  \FunctionTok{colnames}\NormalTok{(tbl) }\OtherTok{=} \FunctionTok{c}\NormalTok{(}\StringTok{"Колко често доверителният интервал за бета1 съдържа истинската стойност?"}\NormalTok{,}
                    \StringTok{"средната дължина на доверителния интервал на базата на 10000 повторения"}\NormalTok{,}
                    \StringTok{"средното на beta1\_est на базата на 10000 повторения"}\NormalTok{)}
  \ControlFlowTok{for}\NormalTok{ (k }\ControlFlowTok{in} \DecValTok{1}\SpecialCharTok{:}\DecValTok{4}\NormalTok{) \{}
\NormalTok{    tbl[k, }\DecValTok{1}\NormalTok{] }\OtherTok{=} \FunctionTok{paste}\NormalTok{(}\FunctionTok{sum}\NormalTok{((beta1 }\SpecialCharTok{\textgreater{}}\NormalTok{ leftConfInt[k,]) }\SpecialCharTok{\&}\NormalTok{ (beta1}\SpecialCharTok{\textless{}}\NormalTok{ rightConfInt[k,])) }\SpecialCharTok{/} \DecValTok{10000} \SpecialCharTok{*} \DecValTok{100}\NormalTok{, }\StringTok{"\%"}\NormalTok{) }
\NormalTok{    tbl[k, }\DecValTok{2}\NormalTok{] }\OtherTok{=} \FunctionTok{mean}\NormalTok{(rightConfInt[k,] }\SpecialCharTok{{-}}\NormalTok{ leftConfInt[k,]) }\CommentTok{\# ({-}beta1\_sd\_err , +beta1\_sd\_err)}
\NormalTok{    tbl[k, }\DecValTok{3}\NormalTok{] }\OtherTok{=} \FunctionTok{mean}\NormalTok{(beta1\_est[k,]) }
\NormalTok{  \}}
\NormalTok{  n }\OtherTok{=} \FunctionTok{paste}\NormalTok{(}\StringTok{"N ="}\NormalTok{,n)}
  \FunctionTok{print}\NormalTok{(n)}
  \CommentTok{\#Чертаем графиките }
  \FunctionTok{graphics}\NormalTok{(beta1\_est)}
  \FunctionTok{print}\NormalTok{(tbl)}
  
\NormalTok{\}}
\end{Highlighting}
\end{Shaded}

\begin{Shaded}
\begin{Highlighting}[]
\CommentTok{\#Пускаме резултатите за различни стойности на n}
\NormalTok{ns }\OtherTok{=} \FunctionTok{c}\NormalTok{(}\DecValTok{30}\NormalTok{, }\DecValTok{50}\NormalTok{, }\DecValTok{100}\NormalTok{, }\DecValTok{500}\NormalTok{)}
\ControlFlowTok{for}\NormalTok{ (n }\ControlFlowTok{in}\NormalTok{ ns) \{}
  \FunctionTok{f3}\NormalTok{(n)}
\NormalTok{\}}
\end{Highlighting}
\end{Shaded}

\begin{verbatim}
## [1] "N = 30"
\end{verbatim}

\includegraphics{Homework2_files/figure-latex/unnamed-chunk-7-1.pdf}
\includegraphics{Homework2_files/figure-latex/unnamed-chunk-7-2.pdf}

\begin{verbatim}
##    Колко често доверителният интервал за бета1 съдържа истинската стойност?
## а) "95.19 %"                                                               
## б) "94.73 %"                                                               
## в) "95.16 %"                                                               
## г) "95.75 %"                                                               
##    средната дължина на доверителния интервал на базата на 10000 повторения
## а) "0.531176440120034"                                                    
## б) "0.539427270027266"                                                    
## в) "0.53069589573476"                                                     
## г) "0.521551212196019"                                                    
##    средното на beta1_est на базата на 10000 повторения
## а) "4.99991065418752"                                 
## б) "4.99939311774202"                                 
## в) "4.99781644806955"                                 
## г) "4.9980185424313"                                  
## [1] "N = 50"
\end{verbatim}

\includegraphics{Homework2_files/figure-latex/unnamed-chunk-7-3.pdf}
\includegraphics{Homework2_files/figure-latex/unnamed-chunk-7-4.pdf}

\begin{verbatim}
##    Колко често доверителният интервал за бета1 съдържа истинската стойност?
## а) "94.87 %"                                                               
## б) "94.99 %"                                                               
## в) "95.13 %"                                                               
## г) "94.72 %"                                                               
##    средната дължина на доверителния интервал на базата на 10000 повторения
## а) "0.399684402796321"                                                    
## б) "0.405494919336682"                                                    
## в) "0.400818797135155"                                                    
## г) "0.393881210351529"                                                    
##    средното на beta1_est на базата на 10000 повторения
## а) "4.99959393921555"                                 
## б) "4.99977637150879"                                 
## в) "4.99978409578423"                                 
## г) "5.00055948383393"                                 
## [1] "N = 100"
\end{verbatim}

\includegraphics{Homework2_files/figure-latex/unnamed-chunk-7-5.pdf}
\includegraphics{Homework2_files/figure-latex/unnamed-chunk-7-6.pdf}

\begin{verbatim}
##    Колко често доверителният интервал за бета1 съдържа истинската стойност?
## а) "94.74 %"                                                               
## б) "95.22 %"                                                               
## в) "94.98 %"                                                               
## г) "95.3 %"                                                                
##    средната дължина на доверителния интервал на базата на 10000 повторения
## а) "0.278255740332692"                                                    
## б) "0.281296728358233"                                                    
## в) "0.280071194640506"                                                    
## г) "0.276488180845011"                                                    
##    средното на beta1_est на базата на 10000 повторения
## а) "4.99979175510166"                                 
## б) "5.00076798789569"                                 
## в) "4.99816697919449"                                 
## г) "5.00010297435336"                                 
## [1] "N = 500"
\end{verbatim}

\includegraphics{Homework2_files/figure-latex/unnamed-chunk-7-7.pdf}
\includegraphics{Homework2_files/figure-latex/unnamed-chunk-7-8.pdf}

\begin{verbatim}
##    Колко често доверителният интервал за бета1 съдържа истинската стойност?
## а) "94.84 %"                                                               
## б) "95.21 %"                                                               
## в) "95.08 %"                                                               
## г) "94.85 %"                                                               
##    средната дължина на доверителния интервал на базата на 10000 повторения
## а) "0.122486181584946"                                                    
## б) "0.123843516533864"                                                    
## в) "0.123635077941145"                                                    
## г) "0.122408615167866"                                                    
##    средното на beta1_est на базата на 10000 повторения
## а) "5.00041515952469"                                 
## б) "4.99975553929257"                                 
## в) "5.00002024761973"                                 
## г) "5.00023632876921"
\end{verbatim}

\end{document}
